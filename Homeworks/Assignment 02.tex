\documentclass[12pt]{article}
\usepackage{amsmath, amssymb, amsthm}
\usepackage{latexsym, epsfig, ulem, cancel, multicol, hyperref}
\usepackage{graphicx, tikz, subfigure,pgfplots}
\usepackage{blindtext}
\usepackage[a4paper, total={6in, 8in}]{geometry}
\setlength{\parindent}{0pt}
\usepackage{multirow}
\usepackage{mathtools}
\pgfplotsset{width=10cm,compat=1.9}
\usepackage{amsmath, amssymb, amsthm, graphicx, hyperref}
\usepackage{enumerate}
\usepackage{fancyhdr}
\usepackage{multirow, multicol}
\usepackage{tikz}
\usepackage{comment}
\setlength{\parskip}{1ex}

\newcommand{\T}[0]{\top}
\newcommand{\F}[0]{\bot}
\newcommand{\liminfty}[1]{\lim_{#1 \to \infty}}
\newcommand{\limzero}[1]{\lim_{#1 \to 0}}
\newcommand{\limto}[1]{\lim_{#1}}
\newcommand{\Z}{\mathbb{Z}}
\newcommand{\R}{\mathbb{R}}
\newcommand{\C}{\mathbb{C}}
\newcommand{\Q}{\mathbb{Q}}
\newcommand{\odd}[0]{\mathbb{Z} - 2\mathbb{Z}}
\newcommand{\lineint}[1]{\int_{#1}}
\newcommand{\pypx}[2]{\frac{\partial #1}{\partial #2}}
\newcommand{\divg}{\nabla \cdot}
\newcommand{\curl}{\nabla \times}
\newcommand{\dydx}[2]{\frac{d #1}{d #2}}
\newcommand{\sqbkt}[1]{\left[ #1 \right]}
\newcommand{\paren}[1]{\left( #1 \right)}
\newcommand{\tribkt}[1]{\left< #1 \right>}
\newcommand{\abso}[1]{\left|#1 \right|}
\newcommand{\zero}{\{0\}}
\newcommand{\then}{\rightarrow}
\newcommand{\nonneg}{\Z^+ \cup \{0\}}
\DeclarePairedDelimiter\ceil{\lceil}{\rceil}
\DeclarePairedDelimiter\floor{\lfloor}{\rfloor}
\newcommand{\union}[2]{\bigcup_{#1}^{#2}}
\newcommand{\inter}[2]{\bigcap_{#1}^{#2}}
\newcommand{\openclose}[1]{\left( #1 \right]}
\newcommand{\closeopen}[1]{\left[ #1 \right)}
\newcommand{\compo}[2]{#1 e^{i #2}}
\newcommand{\laplase}{\bigtriangleup}

\newtheorem*{remark}{Remark}
\title{\textbf{Applied PDE Assignment 1}}
\author{Dennis Li}
\begin{document}
\maketitle


\begin{enumerate}
    \item[5.1.2] Consider a metal rod \( 0 < x < l \), insulated along its sides but not at its ends, which is initially at temperature \( = 1 \). Suddenly both ends are plunged into a bath of temperature \( = 0 \). Write the differential equation, boundary conditions, and initial condition. Write the formula for the temperature \( u(x, t) \) at later times. In this problem, assume the infinite series expansion:
    
    \[
    1 = \frac{4}{\pi} \left( \sin\left( \frac{\pi x}{l} \right) + \frac{1}{3} \sin\left( \frac{3 \pi x}{l} \right) + \frac{1}{5} \sin\left( \frac{5 \pi x}{l} \right) + \cdots \right)
    \]

    \item[5.1.3] A quantum-mechanical particle on the line with an infinite potential outside the interval \( (0, l) \) (a "particle in a box") is given by Schrödinger's equation \( u_t = i u_{xx} \) on \( (0, l) \) with Dirichlet conditions at the ends. Separate the variables and use (8) to find its representation as a series.






    
\end{enumerate}





\end{document}