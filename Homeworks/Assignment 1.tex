\documentclass[12pt]{article}
\usepackage{amsmath, amssymb, amsthm}
\usepackage{latexsym, epsfig, ulem, cancel, multicol, hyperref}
\usepackage{graphicx, tikz, subfigure,pgfplots}
\usepackage{blindtext}
\usepackage[a4paper, total={6in, 8in}]{geometry}
\setlength{\parindent}{0pt}
\usepackage{multirow}
\usepackage{mathtools}
\pgfplotsset{width=10cm,compat=1.9}
\usepackage{amsmath, amssymb, amsthm, graphicx, hyperref}
\usepackage{enumerate}
\usepackage{fancyhdr}
\usepackage{multirow, multicol}
\usepackage{tikz}
\usepackage{comment}
\setlength{\parskip}{1ex}

\newcommand{\T}[0]{\top}
\newcommand{\F}[0]{\bot}
\newcommand{\liminfty}[1]{\lim_{#1 \to \infty}}
\newcommand{\limzero}[1]{\lim_{#1 \to 0}}
\newcommand{\limto}[1]{\lim_{#1}}
\newcommand{\Z}{\mathbb{Z}}
\newcommand{\R}{\mathbb{R}}
\newcommand{\C}{\mathbb{C}}
\newcommand{\Q}{\mathbb{Q}}
\newcommand{\odd}[0]{\mathbb{Z} - 2\mathbb{Z}}
\newcommand{\lineint}[1]{\int_{#1}}
\newcommand{\pypx}[2]{\frac{\partial #1}{\partial #2}}
\newcommand{\divg}{\nabla \cdot}
\newcommand{\curl}{\nabla \times}
\newcommand{\dydx}[2]{\frac{d #1}{d #2}}
\newcommand{\sqbkt}[1]{\left[ #1 \right]}
\newcommand{\paren}[1]{\left( #1 \right)}
\newcommand{\tribkt}[1]{\left< #1 \right>}
\newcommand{\abso}[1]{\left|#1 \right|}
\newcommand{\zero}{\{0\}}
\newcommand{\then}{\rightarrow}
\newcommand{\nonneg}{\Z^+ \cup \{0\}}
\DeclarePairedDelimiter\ceil{\lceil}{\rceil}
\DeclarePairedDelimiter\floor{\lfloor}{\rfloor}
\newcommand{\union}[2]{\bigcup_{#1}^{#2}}
\newcommand{\inter}[2]{\bigcap_{#1}^{#2}}
\newcommand{\openclose}[1]{\left( #1 \right]}
\newcommand{\closeopen}[1]{\left[ #1 \right)}
\newcommand{\compo}[2]{#1 e^{i #2}}
\newcommand{\laplase}{\bigtriangleup}

\newtheorem*{remark}{Remark}
\title{\textbf{Applied PDE Assignment 1}}
\author{Dennis Li}
\begin{document}
\maketitle
\section{Section 1.1}
2. Which of the following operators are linear?
\begin{itemize}
    \item[(a)] $\mathcal{L} u = u_x + x u_y$ 
    
    This is linear.
    \item[(b)] $\mathcal{L} u = u_x + u u_y$ 
    
    This is not linear.
    \item[(c)] $\mathcal{L} u = u_x + u_y^2$ 
    
    This is not linear.
    \item[(d)] $\mathcal{L} u = u_x + u_y + 1$ 
    
    This is linear but not homogeneous.
    \item[(e)] $\mathcal{L} u = \sqrt{1 + x^2} (\cos y) u_x + u_{xy} - [\arctan(x/y)] u$ 

    This is Linear.
\end{itemize}

3. For each of the following equations, state the order and whether it is nonlinear, linear inhomogeneous, or linear homogeneous; provide reasons.
\begin{itemize}
    \item[(a)] $u_t - u_{xx} + 1 = 0$

    This is linear but not homogeneous, since the unknown variable $u$ is raised the power of 1, but $u=0$ is not a solution.
    \item[(b)] $u_t - u_{xx} + xu = 0$

    This is linear and homogeneous, since the unknown variable $u$ is raised the power of 1, and $u=0$ is a solution.
    \item[(c)] $u_t - u_{xx} + uu_x = 0$

    This is not linear but homogeneous, since the the unknown variable $u$ is raised not to the power of 1, but $u=0$ is a solution.
    \item[(d)] $u_{tt} - u_{xx} + x^2 = 0$

    This is linear but not homogeneous, since the unknown variable $u$ is raised the power of 1, but $u=0$ is not a solution.
    \item[(e)] $iu_t - u_{xx} + u/x = 0$

    This is linear and homogeneous, since the unknown variable $u$ is raised  power of 1, and $u=0$ is a solution.
    \item[(f)] $u_x \left(1 + u_x^2\right)^{-1/2} + u_y \left(1 + u_y^2\right)^{-1/2} = 0$

    This is not linear but homogeneous, since the the unknown variable $u$ is raised to not the power of 1, but $u=0$ is a solution.
    \item[(g)] $u_x + e^y u_y = 0$

    This is linear and homogeneous, since the unknown variable $u$ is not raised more than the power of 1, and $u=0$ is a solution.
    \item[(h)] $u_t + u_{xxxx} + \sqrt{1 + u} = 0$

    This is not linear and not homogeneous, since the unknown is not raised to the power of 1, and $u=0$ is not a solution. 
\end{itemize}

4. Show that the difference of two solutions of an inhomogeneous linear equation $\mathcal{L} u = g$ with the same $g$ is a solution of the homogeneous equation $\mathcal{L} u = 0$.
\begin{proof}
    let $v,w$ be some solutions of $\mathcal{L} u = g$. Then by definition, $\mathcal{L} v = g$ and $\mathcal{L} w = g$. If we subtract them, we have $\mathcal{L} v - \mathcal{L} w = g - g = 0$. By definition of a linear operator, $\mathcal{L} v - \mathcal{L} w = \mathcal{L}\paren{v-w}=0$. Therefore $v-w$ is a solution to $\mathcal{L} u = 0$.
\end{proof}

\newpage
\text{11.} Verify that \( u(x, y) = f(x) g(y) \) is a solution of the PDE \( uu_{xy} = u_x u_y \), for all pairs of (differentiable) functions \( f \) and \( g \) of one variable.
    \begin{proof}
        Let$f(x) g(y)$ be any differentiable functions of $x,y$, and $u(x,y) = f(x)g(y)$. We can find the following
        \[
        u_{xy} = \pypx{^2}{x\partial y}u(x,y) = \pypx{}{y}f'(x)g(y) = f'(x)g'(y) 
        \]
        \[
        u_x = f'(x)g(y) \quad u_y = f(x)g'(y)
        \]
        Then
        \[
        uu_{xy} = f(x) g(y)f'(x)g'(y)  =f'(x)g(y)f(x)g'(y) =  u_xu_y 
        \]
    \end{proof}

12. Verify by direct substitution that
\[
u_n(x,y) =\sin(nx)\sinh(ny)
\]
is a solution of $\nabla^2 u = u_{xx} + u_{yy} = 0$ for every $n>0$.
\begin{proof}
    First we find $u_{xx}$
    \[
    u_{xx} = \partial_x \paren{n\cos(nx)\sinh(ny)} = -n^2\sin(nx)\sinh(ny)
    \]
    And then we find $u_{yy}$
    \[
    u_{yy} = \partial_y\paren{n\sin(nx)\cosh(ny)} = n^2\sin(nx)\sinh(ny)
    \]
    And we can see that
    \[
    u_{xx} + u_{yy} = 0
    \]
    Therefore $u_n(x,y) =\sin(nx)\sinh(ny)$ is a solution to this PDE. 
\end{proof}

\section{Section 1.2}
1. Solve the PDE
\[
\begin{cases}
    2u_t + 3u_x = 0\\
    u(x,t=0) = \sin x
\end{cases}
\]
We start by defining a change of variable
\[
t' = 2t + 3x \quad x' = 3t-2x
\]
And we can perform the following operation
\[
u_t = \pypx{u}{t} = \pypx{u}{t'}\pypx{t'}{t} + \pypx{u}{x'}\pypx{x'}{t} = 2u_{t'} + 3u_{x'}
\]
\[
u_x = \pypx{u}{x} = \pypx{u}{x'}\pypx{x'}{t} + \pypx{u}{t'}\pypx{t'}{x} = 3u_{t'} - 2u_{x'}
\]
Therefore our original equation becomes
\[
2\paren{2u_{t'} + 3u_{x'}} + 3\paren{3u_{t'} - 2u_{x'}} = 0
\]
This simplifies to
\[
u_{t'} = 0
\]
We divided both side by $13$ since $13 \neq 0$. This means that the the function has no component of $t'$, therefore the general solution is $u(t,x) = A(x') = A(3t-2x)$. Now we use the initial value $u(x,t=0) = \sin x$ to find the particular solution.
\[
u(x,t=0) = A(-2x) = \sin x
\]
Let $\omega = -2x$, then we have $x = -\frac{\omega}{2}$
\[
A(\omega) = \sin \paren{\frac{\omega}{2}}
\]
And the particular solution is
\[
u(t,x) = \sin\paren{\frac{3t-2x}{2}}
\]


2. Solve the PDE for general solution
\[
3u_y + u_{xy} = 0
\]
Let $\nu = u_y$, we have
\[
3\nu + \nu_x = 0
\]
This has simplified to an ODE. First we can rewrite it as
\[
\dydx{\nu}{x} = -3 \nu
\]
And perform a separation of variable
\[
\frac{d\nu}{\nu} = -3dx
\]
Integrating both side with respect to $x$, we have
\[
\ln\abso{\nu} = -3x+C(y)
\]
Raising $e$ to RHS and LHS, and with some simplification, we have
\[
\nu = A(y)e^{-3x} \quad A(y) = \pm e^{C(y)}
\]
Since $\nu = u_y$, we can integrate both side with respect to $y$, and obtain
\[
u(x,y) = \paren{\int A(y)\;dy}e^{-3x}
\]

\end{document}