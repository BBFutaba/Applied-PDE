\documentclass[12pt]{book}
\usepackage{amsmath, amssymb, amsthm}
\usepackage{latexsym, epsfig, ulem, cancel, multicol, hyperref}
\usepackage{graphicx, tikz, subfigure,pgfplots}
\usepackage{blindtext}
\usepackage[a4paper, total={6in, 8in}]{geometry}
\setlength{\parindent}{0pt}
\usepackage{multirow}
\usepackage{mathtools}
\pgfplotsset{width=10cm,compat=1.9}
\usepackage{amsmath, amssymb, amsthm, graphicx, hyperref}
\usepackage{enumerate}
\usepackage{fancyhdr}
\usepackage{multirow, multicol}
\usepackage{tikz}
\usepackage{comment}
\setlength{\parskip}{1ex}

\newcommand{\T}[0]{\top}
\newcommand{\F}[0]{\bot}
\newcommand{\liminfty}[1]{\lim_{#1 \to \infty}}
\newcommand{\limzero}[1]{\lim_{#1 \to 0}}
\newcommand{\limto}[1]{\lim_{#1}}
\newcommand{\Z}{\mathbb{Z}}
\newcommand{\R}{\mathbb{R}}
\newcommand{\C}{\mathbb{C}}
\newcommand{\Q}{\mathbb{Q}}
\newcommand{\odd}[0]{\mathbb{Z} - 2\mathbb{Z}}
\newcommand{\lineint}[1]{\int_{#1}}
\newcommand{\pypx}[2]{\frac{\partial #1}{\partial #2}}
\newcommand{\divg}{\nabla \cdot}
\newcommand{\curl}{\nabla \times}
\newcommand{\dydx}[2]{\frac{d #1}{d #2}}
\newcommand{\sqbkt}[1]{\left[ #1 \right]}
\newcommand{\paren}[1]{\left( #1 \right)}
\newcommand{\tribkt}[1]{\left< #1 \right>}
\newcommand{\abso}[1]{\left|#1 \right|}
\newcommand{\zero}{\{0\}}
\newcommand{\then}{\rightarrow}
\newcommand{\nonneg}{\Z^+ \cup \{0\}}
\DeclarePairedDelimiter\ceil{\lceil}{\rceil}
\DeclarePairedDelimiter\floor{\lfloor}{\rfloor}
\newcommand{\union}[2]{\bigcup_{#1}^{#2}}
\newcommand{\inter}[2]{\bigcap_{#1}^{#2}}
\newcommand{\openclose}[1]{\left( #1 \right]}
\newcommand{\closeopen}[1]{\left[ #1 \right)}
\newcommand{\compo}[2]{#1 e^{i #2}}
\newcommand{\laplase}{\bigtriangleup}

\newtheorem*{remark}{Remark}
\title{\textbf{Applied Partial Differential Equation}}
\author{Dennis Li}
\begin{document}
\maketitle

\tableofcontents
Course Grade Composition is one third midterm, two third final.
Exam questions will be directly from the homework with additional questions that asks you about the understanding of the material.

\chapter{Intro to PDE}

\section{Evolutionary PDEs}
In this section, we focus on finding a function $u(t)$. We are interested in how the function evolves in time. We will primarily ve studying this sort of PDEs.

An example being 
\[
\pypx{}{t} u = v \bigtriangleup u = \nu(\partial_{xx} + \partial_{yy} + \partial_{zz})u
\]
Where $\nu$ is some constant, and $\nu(t,\Vec{x})$ is our dependent variable or unknown. Since it depends on the independent variable, time and position, $t, \Vec{x}$ 

if we are only interested in one dimensional behavior, this simplifies to
\begin{align*}
\pypx{}{t}u = \nu \pypx{^2}{x^2} u 
\end{align*}
or we can use a different notation
\[
u_t = \nu u_{xx}
\]
This is the \textbf{one dimensional heat equation}

\subsection{Change of variable}
say we have $u(x,t)$, we change it to $u(x_+,x_-)$ by the rule that $x_+ = x+t$ and $x_- = x-t$

We want to think about it like $u\paren{x(x_+,x_-),t(x_+,x_-)}$

And we can represent them by $x = \frac{x_++x_-}{2}$ and $t = \frac{x_+-x_-}{2}$

\subsubsection{Example 1.1.1}
The following are examples of linear and homogeneous PDEs. Homogeneous means $u=0$ is a solution.

Every function here has constant coefficient, where the coefficient does not depend on time or space. It is hard to work with those with non-constant coefficient, but we can learn their property by working with constant ones. 

All example below are \textit{local} equations. It means that on both side of the equation, we are working with only with some $u(x,t)$. PDEs are always local, but on the verge of being non-local. But they are almost non-local since if you were to approximate the derivatives, you introduced some non-locality.
\begin{enumerate}
    \item $\partial_t u = vu_{xx}$\\
    This PDE is first order in time, second order in space, we will call this a second order equation.
    \item $\partial u = cu_x$\\
    This PDE is first order in time, first order in space, we will call this a first order equation.
    \item $\partial_{tt}u - c^2u_{xx}$
    This PDe is second order in time, second order in space, we will call this a second order equation.
\end{enumerate}

\subsection{Transport Equation}
Examine the following
\[
\partial_t u + c\partial_x u = 0
\]
We can actually factor the operators
\[
\paren{\partial_t + c\partial_x} u = 0
\]
This is a good fit to do change of variables, where we set $x_+ = x+ct$, and $x_- = x-ct$. Just as we have done before, this is essentially
\[
x = \frac{x_++x_-}{2} \quad t = \frac{x_+-x_-}{2c}
\]
Now we want to use our newly defined variable to change the original equation. To do so, we would need to figure out how to get rid of $\partial_t$ and $\partial_x$. We can look at them individually. 
\[
\pypx{}{t} u = \pypx{x_+}{t}\pypx{u}{x_+} + \pypx{x_-}{t}\pypx{u}{x_-} = c\paren{u_{x_+} - u_{x_-}}
\]
Here, we see that $\pypx{x_+}{t}$ is simply $c$, and $\pypx{x_-}{t}$ is simply $-c$. Similarly, we can do the same process to $\partial_x$
\[
c \pypx{}{x}u = c\paren{u_{x_+} + u_{x_-}}
\]
And therefore
\[
\paren{\pypx{}{t}+c\pypx{}{x}}u = 2cu_{x_+} = 0
\]
And this means that $u_{x_+} = 0$, indicating that means $u$ is not a function of $x_+$. Therefore
\[
u(x_+,x_-) = A(x_-) 
\]
This means that \textbf{any} function with respect to $x_-$ or any $A(x-ct)$ solves this PDE.  And this is the general solution of this partial differential equation.

How would we interpret this solution. This solution tells us some property about the time evolution of all the solution. As time evolves, the entire function moves to the $+x$ direction. This is in fact a wave. 

If there were to be an initial value to be our $u_{init}$, this means that there would be only one solution, with exactly one initial condition. This is the \textbf{initial value problem} in PDE. Basically, the solution tells you how the function transports in time, and the initial data tells us what kind of equation is being transported. 

Now we incorporate the initial value into our PDE problem. 
\[
\begin{cases}
    \partial_t u + c\partial_x u = 0 & t>0,\; -\infty < x < \infty\\
    u(x,t=0)=u_{i}(x) &\forall x \in \R
\end{cases}
\]
We can think of our initial value, an entire function, as an infinite degree information that is needed to nail down the equation. And we can also think of it like follows.
\[
\int_D u_t(t,x) \; dt = u(t,x) + A(x)
\]
Taking an integral of a partial derivative with respect to $t$ leaves you an entire free function of $x$, therefore we would need a function of $x$ as our initial value. 

\subsubsection{Numerical Approximation}
Now we look at the transport equation with initial value. Here, if we have condition
\[
\begin{cases}
    u_t+c\partial_x u = 0 & t>0 \; x\in \R\\
    u(x,t=0) = u_i{x}
\end{cases}
\]
We have solved that $u(x,t) = u_i(x-ct)$. What if we want to approximate the solution without using any calculus. We would first define a small interval on the $x-axis$, we would call it $\delta$. We define $x_j = j\delta$. And the function can now be written as
\[
u(x_j,t) = u(j\delta,t)
\]
The function essentially became a infinite dimensional vector
\[
\mathbf{u}^{N} = \paren{u_1,u_2,\ldots,u_N} \quad \forall j \in \{1,2,\ldots,N\}
\]
And the derivative with respect to $x_j$ will become
\[
\partial_x u(x_j) \simeq \frac{u(x_{j+1})-u(x_j)}{\delta}
\]
And we can solve it like an ODE
\[
\dydx{}{t}u_j = \frac{-c}{\delta}(u_{j+1}-u_j)
\]
Where
\[
u_j(t) = u(x_j,t) = u(j\delta,t)
\]
We have successfully approximate the PDE with a series of ODEs. We can see that the smaller we make the interval, the better the approximation. Therefore it is a good way to think of PDE as an infinite collection of ODEs. 

\subsubsection{Physical Derivation of Transport Equation}
Imagine we have a pipe where water flows through it at speed $c$. The water is contaminated and the pollutant has density as a function of position and time, $u(x,t)$, and has dimensions $g/mm$.  

Now we imagine $x=0$ as the beginning of the pipe, and $x=b$ as the cut off point where we are interested, we define a function $M(b,t)$ to be the amount of pollutant in the section from $x=0 \to x=b$. Or it can be written as
\[
M(b,t) = \int_{0}^{b}u(x,t)\;dx
\]
And after a certain time $\tau$ after time $t$, the amount of pollutant can be characterized with
\[
M(b+c\tau,t+\tau) = \int_{c\tau}^{b+c\tau}u(x,t)\;dx
\]
Where $c\tau$ is the distance the pollutants moved along with the speed of the water in the tube. 
If we think of the blob of pollutant be moving through the pipe with the water, we can model it like the follows. 
\[
M(b,t) = M(b+c\tau,t+\tau)
\]
And sequentially
\[
\int_{0}^{b}u(x,t)\;dx = \int_{c\tau}^{b+c\tau}u(x,t)\;dx
\]
We are basically transporting this blob of stuff in the pipe. If we would take the partial dirivative of this integral with respect to $b$ and $\tau$, we have
\begin{align*}
    u(b,t) = u(b+c\tau,t+\tau)\\
    \left. 0 = u_t + cu_b \right|_{t=0}(b,t)
\end{align*}
We have obtained the transport equation
\[
u_t + c\partial_b u = 0, \;\; u(b,t)
\]

\subsubsection{Physical Derivation of Diffusion Equation}
Now if the pipe has no definitive starting point, and is starting at an arbitrary position $x_0$, and we have the cutoff $x_1$. The pollutant comes in from $x_0$ and leaves at $x_1$. 

If the water is not moving, but the pollutant is spreading out in the water, the physical law that characterize this behavior is called the \textit{Fisch's Law}. And it says
\[
-R \pypx{}{x}u(x,t),\; R > 0
\]
This expression characterize the flow rate of the pollutants.
We define a function that relates the amount of pollutant with only time.
\[
M(t) = \int_{x_0}^{x_1}u_(x,t)\; dx
\]
We can differentiate the equation with respect to $t$, and obtain
\[
M'(t) = \int_{x_0}^{x_1}u_t(x,t)\; dx
\]
And this is equivalent as the flow rate of the pollutant, therefore
\[
M'(t) = \int_{x_0}^{x_1}u_t(x,t)\; dx = -R\partial_{x_0}u(x_0,t) + R\partial_{x_1}u(x_1,t) 
\]
And we would obtain the \textbf{diffusion equation}, or the \textbf{heat equation}.
\begin{align}
u_t(x,t) = Ru_{xx}(x,t)
\end{align}
This would eventually give us 
\[
u_t = c\partial_x u
\]
Which is indeed the transport equation. 



\subsection{Wave Equation}
\[
u_tt - c^2\nabla^2 u =0
\]
Now we look at this partial differential equation
\begin{align}
    u_{tt} - c^2u_{xx} = 0 \quad t>0,\; -\infty<x<\infty
\end{align}
And the initial value will consist of 2 functions of $x$ since this is second order in time. 
\begin{align}
    u(x,t=0)=u_{i}(x)\\
    u_t(x,t=0)=v_{in}(x)
\end{align}
And this is the one dimensional linear wave equation in. Now how do we solve it. We can factor the operators like before
\[
\paren{\partial_{tt}-c^2\partial_{xx}}u = 0
\]
And we can factor it further, treating the second derivative as some sort of a square, and get
\[
\paren{\partial_t + c\partial_x}\paren{\partial_t - c\partial_x}u = 0
\]
And we can use the same change of variable as before, where
\[
x_+ = x+ct \quad x_- = x-ct
\]
We would obtain
\[
\frac{\partial}{\partial x_+}\frac{\partial}{\partial x_-} u = 0
\]
We by observation of this equation, we can see that the solution can be obtained like follows.
\[
\int \frac{\partial}{\partial x_+}\frac{\partial}{\partial x_-} u \; dx_+= A'(x_-) 
\]
and we have
\[
\int \frac{\partial}{\partial x_-} u \; dx_-= \int A'(x_-) \; dx_-
\]
If we define $\dydx{}{x_-}A(x_-)=A'(x_-)$, this would give us
\[
u(x_+,x_-)=A(x_-)+B(x_+)
\]
And if we substitute back, we have
\[
u(x_+,x_-)=A(x-ct)+B(x+ct)
\]
Think of it like the function is composed of one function of $x_-$ and another function of $x_+$, such that when you carry out the sequence of these partial derivatives you get 0. 

We have 2 degrees of infinite solution for $A(x_-)$ and $B(x_+)$. To obtain the precise function that solves certain problems, we will need 2 initial value, or 2 initial function to obtain it. 

So now, we use the 2 functions we received at the beginning as our initial value and solve for the exact solution.  We would plug the solution in to our initial value.
\[
u(x,t=0)=u_{i}(x)
\]
This means that
\[
\left. A(x-ct)+B(x+ct)\right|_{t=0} = A(x)+B(x) = u_{i}(x)
\]
And the other initial value gives us
\[
\left. A'(x-ct)+B'(x+ct)\right|_{t=0} =-cA'(x)+cB'(x) = u_{i}(x)
\]
We can take the derivative of the first equation, and receive the following set of equations
\[
\begin{cases}
    A'(x)+B'(x)=u'_{i}(x)\\
    cB'(x)-cA'(x)=u_{i}(x)
\end{cases}
\]
We can multiply the first equation by $c$ and add two functions. After some manipulation, we will have
\[
B'(x) = \frac{1}{2} u'_{in}(x) + \frac{1}{2c}u_{i}(x)
\]
And we can do the same process and subtract the two equations and get
\[
A'(x) = \frac{1}{2} u'_{i}(x) - \frac{1}{2c}u_{i}(x)
\]
Now we integrate the both of these solutions
\[
A(x-ct)=\int_{-\infty}^{x-ct} A'(y) \; dy = \frac{1}{2}u_{i}(x-ct)-\frac{1}{2c}\int_{-\infty}^{x-ct}u_{i}(y)\;dy
\]
\[
B(x+ct)=\int_{-\infty}^{x+ct} B'(y) \; dx = \frac{1}{2}u_{i}(x+ct)+\frac{1}{2c}\int_{-\infty}^{x+ct}u_{i}(y)\;dy
\]
And we can rewrite the solution
\[
u(x,t) = \frac{1}{2}u_{i}(x-ct)-\frac{1}{2c}\int_{-\infty}^{x-ct}u_{i}(y)\;dy + \frac{1}{2}u_{i}(x+ct)+\frac{1}{2c}\int_{-\infty}^{x+ct}u_{i}(y)\;dy 
\]
We can therefore simplify the integrals, and finally obtain the final solution to the one dimensional wave equation.
\begin{align}
u(x,t) = \frac{1}{2}u_{i}(x-ct)+ \frac{1}{2}u_{i}(x+ct)-\frac{1}{2c}\int_{x-ct}^{x+ct}u_{i}(y)\;dy 
\end{align}
What happened here is we integrated the function from $-\infty$ to a variable we defined, such as $x+ct$ or $x-ct$. This effectively swapped the variable in the integral to something we desire.

We have derived the general formula to solve the wave equation
\[
u(x,t) = \frac{1}{2}\sqbkt{\phi(x+ct)+\phi(x-ct)}+\frac{1}{2c}\int_{x-ct}^{x+ct} \psi(s) \;ds
\]
Where $\phi$ is the function $u(x,t=0)$ and $\psi$ is the function $u_t(x,t=0)$.


\section{Diffusion Equation}
Also called heat equation, will be the center of discussion for this section.
\[
\partial_t u = k\partial_{xx}u
\]
\[
t>0\;\; -\infty < x <\infty \;\; \nu\in +\R
\]
We are looking at the one dimensional diffusion equation. In 3 dimension, we will have
\[
\partial_t u = \nu\nabla^2u
\]
We will currently look at only one dimensional. In terms of initial value, we have to specify
\[
\begin{cases}
    \partial_t u = \nu\partial_{xx}u\\
    t>0\;\; -\infty < x <\infty \;\; \nu\in +\R\\
    u(x,t=0) = u_i(x)
\end{cases}
\]
We are looking at something that is linear and homogeneous. This is important since this allows the linear combination of solutions to also be solutions. Now how to we solve for this equation? We take an educated guess that the solution may take the following form
\begin{large}
\[
u(x,t) = e^{i(kx-\omega t)} \quad k,\omega \in \C
\]
\end{large}
Then we can substitute it back into the diffusion equation, and obtain the following equation
\[
-i\omega e^{i(kx-\omega t)} = -\nu k^2 e^{i(kx-\omega t)}
\]
And the term $e^{i(kx-\omega t)}$ can be canceled. We have
\[
-i\omega = -\nu k^2
\]
After simplification, we have
\[
\omega(k) = -i\nu k^2
\]
Substitute it back in, we have
    \[
    u(x,) = e^{i(rx+i\nu k^2 t)}
    \]
Therefore our final solution to this equation is
\begin{large}
    \[
    u(x,t;k) = e^{ikr}e^{-\nu k^2 t} \quad \forall k \in \C
    \]
\end{large}
How do we interpret this solution. We can first analyze what happens if $k \in \C - \R$. If $k$ has imaginary component, the function either blows up at $-\infty$ or $\infty$. Therefore we only accept real values of $k$. And the actual solution that makes sense is
\begin{large}
    \[
    u(x,t;k) = e^{ikr}e^{-\nu k^2 t} \quad \forall k \in \R
    \]
\end{large}
We now have a collection of solutions that is bounded in space and decays in time. We can also define a special case where we can imagine something diffusing in a closed loop.
\[
u(x+L,t) = u(x,t) \quad L = 2\pi n
\]
Now we are interested in the periodic behaviour of the function. In this case
\[
e^{ikL} = 1 \quad k_j = \frac{2\pi j}{L}
\]

Our most general solution for the diffusion equation is
\begin{large}
    \[
    u_{gen}(x,t) = \sum_{j=-\infty}^{\infty}c_je^{i\sqbkt{k,t}}e^{-\nu k_j^2 t}
    \]
\end{large}
For our periodic case, we have one of the term equals to 1, so we can simplify it
\begin{large}
    \[
u_L(x) = \sum_{j=-\infty}^{\infty}c_j e^{ik_j x}, \quad k_j = \frac{2\pi j}{L}
\]
\end{large}


\section{Fourier Series}
The goal is to approximate some arbitrary function with a bunch of sines and cosines. We state that any function can be represented as
\[
f(x) = \sum A_n\cos\paren{\frac{n\pi x}{l}} + \sum B_n\sin\paren{\frac{n\pi x}{l}}
\]
We assumes it converges and this stuff just works. 

\[
\int_{-L}^{L} \cos\paren{\frac{n\pi x}{L}}\cos\paren{\frac{m\pi x}{L}} \; dx = \begin{cases}
    2L & n=m\\
    L & n = m\neq 0\\
    0 & n\neq m
\end{cases}
\]
\[
\int_{-L}^{L} \sin\paren{\frac{n\pi x}{L}}\sin\paren{\frac{m\pi x}{L}} \;dx = \begin{cases}
    L & n = m\\
    0 & n\neq m
\end{cases}
\]
\[
\int_{-L}^{L} \sin\paren{\frac{n\pi x}{L}}\cos\paren{\frac{m\pi x}{L}} \dx =  0
\]
We arrive at
\[
\int_{-L}^{L} f(x) \cos\paren{\frac{m\pi x}{L}}\; dx = \int_{-L}^{L} \cos\paren{\frac{n\pi x}{L}}\cos\paren{\frac{m\pi x}{L}} \; dx + 0
\]
And we have
\[
\int_{-L}^{L} f(x) \cos\paren{\frac{m\pi x}{L}}\; dx = \begin{cases}
    A_m(2L) & n=m-0\\
    A_m(L) & n=m\neq 0
\end{cases}
\]
And we have to find the coefficient $A_m$
\[
A_m = \frac{1}{L}\int_{-L}^{L}f(x)\cos\paren{\frac{m\pi x}{L}} \; dx 
\]
Where $m\in\nonneg$
\[
B_m = \frac{1}{L}\int_{-L}^{L}f(x)\sin\paren{\frac{n\pi x}{L}} \; dx 
\]
Where $n \in \nonneg$
\[
A_0 = \frac{1}{2L}\int_{-L}^{L} f(x) \; dx 
\]

















\end{document}